\chapter{DESCRIPCION DE LA INVESTIGACIÓN}
\section{Planteamiento/Identificación del problema}
Acceder a un crédito educativo implica para el estudiante que desea financiar su matrícula; primero conocer las diferentes ofertas que tiene el mercado, seleccionar cuál de estas le son más favorables, llenar los requisitos de una o varias de esas ofertas, presentarlas y esperar que una de las entidades le apruebe el crédito, con la incertidumbre que esto genera y el tiempo que puede tomar. Además, sin un buen análisis de un crédito o por falta de información o asesoría, un estudiante puede estar haciéndose de una obligación difícil de cumplir o que va en desventaja de sus intereses, como pasa en algunas ocasiones. 

Conocer los diferentes créditos ofertados y analizar en cuál de estos un estudiante tiene mayores opciones y mejores garantías de cumplimiento, es un proceso desgastante y que desestima. Para mejorar este proceso se propone desarrollar un chatbot que sirva de asesor financiero y que ayude al estudiante en el proceso de contrastación de los requisitos y características de las diferentes opciones de créditos disponibles en el sector financiero, indicando al final cuál sería para él la mejor oferta, la que mejor se ajuste a sus necesidad y posibilidades.
\newpage

\section{Objetivos}
\subsection{Objetivo general}
Desarrollar un analizador financiero que brinde asesoría a través de un Chatbot a los estudiantes universitarios que estén en la búsqueda de un crédito para financiar los costos de su matrícula.

\subsection{Objetivos específicos}
\begin{itemize}
	\item Establecer con un proceso de indagación en varias entidades financieras las constantes y variables de los requisitos, condiciones y características de un crédito educativo, como base de información para ser contrastada con los datos de los estudiantes interesados en financiar su matrícula.
	\item Definir una técnica para el análisis y contrastación de los datos recolectados del estudiante contra las características y condiciones de los créditos educativos.
	\item Implementar la tecnología necesaria para la construcción de la base de motor de diálogo del chatbot y el medio a través del cual realizar el despliegue para alcanzar la interacción con el estudiante.
	\item Definir el proceso de entrenamiento a través del cual se pueda alimentar la base de datos del Chatbot.
\end{itemize}
\newpage

\section{Justificación del trabajo/investigación}
\subsection{Justificación metodológica}
Son muchos los créditos que ofrece el sector financiero para aquellos estudiantes que desean financiar su matrícula académica, cada uno de estos créditos se ofrecen con características que pueden ser similares entre sí o que pueden ser diferentes, como lo es la tasa de interés, el número de cuotas, las posibilidades de renegociación etc. dichas características son para el estudiante variables que deben ser analizadas, contrastadas y asumidas al momento de seleccionar la mejor posibilidad de crédito, un proceso que requiere contar con la suficiente información y con el tiempo necesario para realizarse. Con poco tiempo e información un estudiante puede estar seleccionando y postulándose a un crédito que no se ajusta a sus posibilidades e intereses y se puede convertir esta obligación en una deuda difícil de pagar.

Es debido a esto que se propone el desarrollo de un chatbot que asesore, analice y contraste la información del estudiante contra los requisitos y características de cada una de las ofertas de créditos educativos existentes en el sector financiero, y que al final presente un top de las mejores opciones de créditos que se ajusten a las necesidades y posibilidades de cumplimiento del estudiante.
\newpage

\section{Hipótesis}
El uso por parte de un estudiante universitario de un chatbot que asesore y analice los datos financieros, para agilizar la selección de un crédito, ayuda que este tome una buena decisión al momento de buscar financiación de su matrícula académica y no termine haciéndose de una obligación difícil de cumplir.
\newpage

\section{Marco referencial}
\newpage

\section{Metodología de la investigación}
\newpage

\section{Organización del trabajo de grado}
\newpage

\section{Estudio de sistemas previos}
\newpage